\chapter{序論}\label{chap:introduction}

はじめに、本章では本論文における研究の動機と目的を述べた後、
本論文の構成について概略を述べる。

\newpage

\section{研究動機}

本論文は、近接関係に基づく情報共有と情報取得のためのアプリケーションを提案し、
その設計および開発について述べ、場所や状況を共有する人々のための情報共有を実現する。

アプリケーションはユーザ同士の近接関係から、情報を共有または取得し、またWeb上から情報の付加を行える機能を提供する。

近年の情報社会において、人々がインターネットから情報を得るための
コミュニケーション・プラットフォームはめざましい変化を続けてきた。
情報を扱う端末としてスマートフォンやタブレットなどのモバイルデバイスが普及しており、
人々はモバイルデバイスからあらゆる情報を共有することが可能な社会になっている。
インターネット上では時間および空間に縛られないコミュニケーションが発達してきた。
近年発展を遂げているソーシャルネットワークサービスでは、遠方にいる人物だろうと問題にされず、
情報が共有されるか否かは人と人との関係性によって区別される。

このようなインターネットからの情報は、情報と発信者の持つコンテキストは考慮されず、
あらゆる情報が入り混じっている。
その中から現在の自分のコンテキストに合った情報を抜き取るのは難しい。

近年モバイルデバイスが持つ機能の進化につれて、
デバイスが持つセンサやモジュールを利用したサービスアプリケーションが一般的なものとなってきた。
デバイスに組み込まれている様々なセンサの値は、情報と実世界を結びつける値としてサービスに利用されている。
このようなContext-basedなサービスアプリケーションは、
人間同士により強固な親近感を持った繋がりを持たせることが可能となっている。

人と人が面と向かって情報を共有する際は、両者が同じ場所に居合わせているというコンテキストを持っている。
同じ生活圏やイベントなどでも、そこに集まる人々と同じ状況を共有しているという
コンテキストを持っているのにもかかわらず、人々の持つ情報は共有されないまますれ違っている。
人々が端末から情報を共有し合うとき、たとえ同じ空間にいる者同士でも、
その場に居合わせているというコンテキストは考慮されることなく行われている。

このコンテキストを有効に扱い、サービスに利用することが可能ならば、
煩雑な事前情報のやりとりなく、状況や場を共有する人間同士の情報共有は活発になることだろう。

本論文では、この世界を流れている膨大な情報の中から、
自分たちが居合わせている状況や場所、そしてそこにいる人々にとって有用となる情報を選別し、
個人にとってその場所で有用な情報を取得するためのアプリケーションを提案する。


\newpage


\section{本論文の目的}

本論文では、近接関係を利用した情報共有、および情報取得の手法について提案し、設計及び開発を行った後、
その有用性について明らかにすることを目的とする。


\section{本論文の構成}

第\ref{chap:background}章では、コンテキストを共有する人々に向けたアプリケーションの現在の状況、およびその中の
場所や状況を共有する人々のためのアプリケーションの背景について述べる。

第\ref{chap:design}章、第\ref{chap:implementation_1}章、第\ref{chap:hyoka}章では、
場所や状況を共有する人間が情報を共有するためのアプリケーション「そこにいる名無しさん」
の設計思想、実装および評価について述べる。

第\ref{chap:implementation_2}章では、「そこにいる名無しさん」の補助となるよう作られたツールについて述べる。

第\ref{chap:application}章では、本論文で提案するアプリケーションによって実現する情報共有手法、
および応用例について述べる。

第\ref{chap:related}章では、関連する研究分野、および既存のサービスやアプリケーションを紹介する。

第\ref{chap:discussion}章では、本論文で提案されてきたものに対する考察を行う。

第\ref{chap:conclusion}章では、本論文の成果をまとめ、今後の展望について述べる。
