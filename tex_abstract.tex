% ■ アブストラクトの出力 ■
%	◆書式:
%		begin{jabstract}〜end{jabstract}	:日本語のアブストラクト
%		begin{eabstract}〜end{eabstract}	:英語のアブストラクト
%		※ 不要ならばコマンドごと消せば出力されない。



% 日本語のアブストラクト
\begin{jabstract}

モバイルデバイスの進化につれて、
一般的なモバイルデバイスに共通して搭載されているセンサやモジュールは豊富になった。
同時に、それらと組み合わせることで実現されるContext-basedなサービスも一般化している。
これらのサービスはコンテキストに特化した情報の取得を促進させつつも、
情報はサービス内限定の閉鎖的な文脈となることが多い。

対して、近年大きく発展してきたSNSやコミュニケーションサービスは、
Webの持つ自由度を持ったまま進化を続けてきた。
しかし、同じ場所に居合わせた人同士がこのようなサービスの上で情報をやりとりするには
あらかじめユーザが互いの信頼において連絡方法を確立しなくてはならない。
そのため、見知らぬ人同士がデジタルな情報共有まで辿り着くには、壁が存在する。

本論文では、
近接した人同士が状況や場所に関する情報を取得し、共有するためのシステムを提案する。

本システムは、ユーザの近接関係を元にした情報取得と共有を促進させ、
さらにWeb上での閲覧と編集を可能にすることで、コンテキストに対するWebからの情報取得を実現する。
Context-basedなサービスとWebサービスの特性を組み合わせることで、
近接した人同士が情報をやりとりする行為を簡便なものへと発展させつつ、
Context-basedなサービスの閉鎖性を緩和させる。

本論文では、提案するシステムの実装と試用の際の議論から、システムの有効性について考察し、
近接関係に基づく情報取得の展望を示す。


\end{jabstract}



% 英語のアブストラクト
\begin{eabstract}

Many sensors are installed on recent mobile devices, and various
context-based Web services are available using such sensor modules.

These services promote the gathering of information that specializes in context,
but this information is often a closed context in the service.

In contrast, in recent years large development to have the SNS and communication services,
It has continued to evolve while holding the degree of freedom with the Web.

However, people with each other, which happened to be the same place to exchange information on such services
Beforehand users must be established how to contact in user's trust.

Therefore, the wall is present to up until arrive at strangers to share users digital information.

We propose a sharing information system for people
with in close proximity to each other to obtain information about the situation or location.

This system is to promote the sharing and gathering information
that is based on the proximity of the users.

And by enabling editing and viewing on the Web,
to achieve the information obtained from the Web for the context.

By combining the characteristics of the Context-based services and Web services,
and while developing the users in close proximity
to communicate information to the simple ones,
thereby relieving the closing of the Context-based services.

In this paper,
from the discussion of the implementation and testing of the proposed system,
we consider the effectiveness of the system,
and show the outlook of information gathering system based on proximity.

\end{eabstract}
