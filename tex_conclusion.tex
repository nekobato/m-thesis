\chapter{結論}\label{chap:conclusion}

本章では、本論文で提案されたアプリケーションの成果を確認したのち、
展望について述べ、本論文を統括する。

\newpage

\section{研究成果}

本論文では、同じコンテキストを持った人同士を繋ぐサービスについて紹介し、
同じ場所や状況を共有する人々の間で行う情報共有についての問題を整理した上で、
その場への情報共有手法としてBluetoothとモバイルデバイス、Webの連携を行ったサービスを提案した。
また、提案からプロトタイプ「そこにいる名無しさん」を設計・実装し、
そのフィードバックから改良を重ねたアプリケーションから共有手法の有効性の考察を行った。

\section{展望}

「そこにいる名無しさん」の設計では、その場で共有された情報をWeb上で閲覧できる仕組みを構築した。
偶然その場に居合わせた人同士がデジタルな情報を共有する手法は、
情報の信頼性の向上と、空間的に適切な有効範囲を探っていく必要がある。
また、今回はBluetooth2.1のプロトコルを利用したシステムの提案を行ったが、
Bluetooth LEによる、より電池消費が少なく柔軟な検出方法も含め、
より適切な近接関係の検出手法を検討していく。

Web上からアクセスできる情報として、現在の実装はアクティビティごとのページ生成となっているが、
更に状況の空気感と言えるものも閲覧できるように、まとめのようなページも生成できるようにできるとなお良いと言える。
そのためにも、アクティビティの中に場の空気を感じられるようなインターフェイスを探っていく必要もある。


\section{結論}

近接デバイスの検出手段としてのBluetoothは、
一般的なモバイルデバイスのモジュールの中でも安価で手に入り、
電池の消費量の面でも非常に優秀なセンサとして活用することができる。
このモジュールの利用とWebの組み合わせによって、どんな状況でもその場に居合わせたというコンテキストを証明することが可能となった。

近接するモバイルデバイスの検出という面でBluetoothは強力な効果を持っており、
これからも、ユーザのコンテキストを読み取る手段としてのBluetoothモジュールはその有効性を示していくだろう。

また、そこからモバイルデバイスに含まれる機能と組み合わせることで、
その場への情報共有手法としてのモバイルデバイスの活用はより活発に、よりユーザの目的に沿った形で実現されると思われる。

情報共有手法としてのアプリケーションである本提案により、
近接関係の検出をベースにした同じ場に偶然居合わせた人同士の共有手法と、
その場を離れても情報の共有を継続できる手法を提示できたと考えている。
